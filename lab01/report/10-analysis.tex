\section{Аналитический раздел}

\subsection{Задача поиска ассоциативных правил}

Правило ассоциации состоит из двух частей, предшествующей и последующей. Предшествующая задача -- это элемент, находящийся в данных. А последующая -- это элемент или множество элементов, которые встречаются в сочетании с предшествующей задачей. \cite{sr1}

В интеллектуальном анализе данных правила ассоциации являются полезными и помогают спрогнозировать поведение клиента.

Для оценки качества полученных рекомендаций используются следующие метрики \cite{sr1}:

\begin{itemize}
	\item Поддержка -- позволяет узнать, в какой части покупательских корзин содержатся все элементы того или иного ассоциативного правила. Определяется как $support(A \rightarrow B) = P(A \cup B)$
	\item Достоверность -- показывает, насколько хорошим является правило для предсказания правой части,когда условие слева верно. Определяется как $confidence(A \rightarrow B) = \frac{P(A \cup B)}{P(A)}$
	\item Интерес -- измеряет силу правила, сравнивая полное правило с предположенной правой частью и рассчитывается, как отношение достоверности правила к частоте появления следствия -- $lift(A \rightarrow B) = \frac{P(A \cup B)}{P(A)P(B)}$
\end{itemize}

\subsection{Apriori}

Данный алгоритм основан на поиске в ширину, в котором свойство того, что с ростом набора поддержка монотонно убывает, позволяет уменьшить объем вычислений.

Принцип работы алгоритма \cite{sr1}:
\begin{enumerate}
	\item[1.] \textbf{Генерация кандидатов} -- алгоритм начинается с создания набора всех возможных одиночных элементов и определения их частоты в данных. Данные элементы называются ``кандидатами'';
	\item[2.] \textbf{Поиск подмножеств} -- далее следует генерация кандидатов более высокого уровня, используя информацию о частоте 1-элементных наборов. Создаются новые наборы элементов, добавляя один элемент к уже существующим кандидатам, которые являются кандидатами следующего уровня;
	\item[3.] \textbf{Оценка поддержки} -- для каждого кандидата подсчитывается частота его появления в транзакциях. Если частота кандидата превышает заданный порог поддержки, то он считается частым и переходит на следующий уровень, иначе -- отбрасывается;
	\item[4.] \textbf{Сбор ассоциативных правил} -- после завершения генерации кандидатов, с использованием частых наборов элементов создаются ассоциативные правила. Для каждого частого набора элементов создаются все возможные комбинации элементов внутри набора для нахождения ассоциативных правил;
	\item[5.] \textbf{Оценка уверенности} -- н данном этапе для каждого ассоциативного правила вычисляется уровень уверенности. Ассоциативные правила с уверенностью выше определенного порога считаются интересными.
\end{enumerate}

\subsection{ECLAT}

Данный алгоритм, в отличие от Apriori, работает на основе более эффективного и компактного представления данных.

Принцип работы алгоритма \cite{sr1}:
\begin{enumerate}
	\item[1.] \textbf{Создание вертикальной структуры данных} -- в отличие от Apriori, который работает с горизонтальной структурой данных, ECLAT использует вертикальную структуру данных. Это означает, что для каждого элемента данных создается список транзакций, в которых этот элемент присутствует;
	\item[2.] \textbf{Рекурсивный поиск} -- ход алгоритма начинается с 1-элементных наборов и проверяется, сколько транзакций содержит каждый элемент. Элементы, удовлетворяющие минимальному порогу поддержки считаются частыми наборами;
	\item[3.] \textbf{Объединение наборов} -- далее следует объединение частых 1-элементных наборов, чтобы создать более крупные наборы элементов. Это происходит путем пересечения вертикальных файлов элементов, которые входят в эти наборы. При этом также проверяется, удовлетворяют ли полученные наборы минимальному порогу поддержки;
	\item[4.] \textbf{Рекурсивное продолжение} -- затем рекурсивно продолжают создаваться все большие наборы элементов до тех пор, пока не будет достигнут максимальный размер набора или не будут удовлетворены пороги поддержки;
	\item[5.] \textbf{Сбор ассоциативных правил} -- после того, как все частые наборы элементов созданы, ECLAT может быть использован для извлечения ассоциативных правил, аналогично Apriori, причем ассоциативные правила определяются на основе уверенности.
\end{enumerate}

\subsection{FP-Growth}

Данный алгоритм представляет собой эффективный и масштабируемый способ нахождения частых наборов элементов, используя структуру данных, называемую FP-деревом.

FP-дерево -- компактная и эффективная структура данных, представляющая собой древовидную структуру, где каждый узел представляет элемент данных, а ребра между узлами -- это связи между элементами в транзакциях. Каждый путь от корня до листа в дереве представляет одну из транзакций из исходных данных, а счетчики на узлах отражают частоту встречаемости элементов. \cite{fpgrowth}

Принцип работы алгоритма \cite{fpgrowth}:
\begin{enumerate}
	\item[1.] \textbf{Построение FP-дерева}:
		\begin{itemize}
			\item Подсчет частоты встречаемости каждого элемента в транзакциях и сортировка элементов по убыванию частоты, таким образом более частые элементы находятся ближе к корню дерева;
			\item Создание корневого узла дерева;
			\item Для каждой транзакции создается путь в дереве, начиная с корневого узла и добавляя элементы транзакции по мере прохождения по дереву. Если элемент уже существует на пути, увеличивается счетчик этого элемента. Если элемент отсутствует -- он добавляется как новый узел в дереве;
		\end{itemize}
	\item[2.] \textbf{Создание условных FP-деревьев}: для каждого элемента, начиная с самого частого, строится условное дерево. Данное дерево создается путем удаления всех путей в FP-дереве, которые не содержат данный элемент, а затем обновления счетчиков элементов на оставшихся путях;
	\item[3.] \textbf{Рекурсивный поиск частых наборов}: для каждого условного дерева рекурсивно находятся все частые наборы элементов, начиная с элементов, которые находятся ближе к корню дерева. Это позволяет извлечь частые наборы элементов, учитывая их иерархию в FP-дереве;
	\item[4.] \textbf{Сбор ассоциативных правил}: после того, как все частые наборы элементов найдены, ассоциативные правила определяются на основе уверенности.
\end{enumerate}
\pagebreak