\section{Технологический раздел}

В данном разделе описываются средства разработки программного обеспечения.

\subsection{Средства реализации}

В качестве используемого был выбран язык программирования Python \cite{Python}.

Данный выбор обусловлен следующими факторами:
\begin{itemize}
	\item Большое количество исчерпывающей документации;
	\item Широкий выбор доступных библиотек для разработки;
	\item Простота синтаксиса языка и высокая скорость разработки.
\end{itemize} 

При написании программного продукта использовалась среда разработки Visual Studio Code. Данный выбор обсуловлен тем, что данная среда распространяется по свободной лицензии, поставляется для конечного пользователя с открытым исходным кодом, а также имеет большое число расширений, ускоряющих разработку.

\subsection{Библиотеки}

При анализе и обработке датасета, а также для решения поставленных задач использовались библиотеки:
\begin{itemize}
	\item pandas;
	\item numpy;
	\item matplotlib;
	\item apyory \cite{apyori};
	\item pyECLAT \cite{pyECLAT};
	\item fpgrowth-py \cite{fpgrowth_py}.
\end{itemize}

Данные библиотеки позволили полностью покрыть спектр потребностей при выполнении работы.