\section{Аналитический раздел}

\subsection{Коллаборативная фильтрация}

Коллаборативная фильтрация -- это метод рекомендации, при котором анализируется только реакция пользователей на объекты -- оценки, которые выставляют пользователи. \cite{collabFilter}

Оценки могут быть как явными, так и неявными (например, длительность нахождения пользователя на странице товара). Целью фильтрации является предсказание оценки пользователем по оценкам других. Чем больше имеется оценок, тем точнее получатся рекомендации. \cite{collabFilter} 

\subsection{Коллаборативная фильтрация по пользователю}

В данном методе предполагается, что пользователи, которые в прошлом имели похожие предпочтения, будут иметь похожие предпочтения и в будущем. Для построения рекомендаций с использованием этого метода выполняются следующие шаги:

\begin{enumerate}
	\item[1.] Создание матрицы, где строки представляют пользователей, а столбцы объекты; значения в ячейках матрицы отражают оценки или действия пользователей по отношению к объектам;
	\item[2.] Вычисление сходства между пользователями (обычно используются косинусное сходство или корреляция Пирсона);
	\item[3.] Для конкретного пользователя генерируются рекомендации на основе сходства с другими пользователями через агрегацию оценок или действий похожих пользователей.
\end{enumerate}

\subsection{Коллаборативная фильтрация по объекту}

В данном методе предполагается, что пользователи будут интересоваться теми объектами, которые похожи на уже положительно оцененные объекты. Для построения рекомендаций с использованием метода выполняются следующие шаги:

\begin{enumerate}
	\item[1.] Создание матрицы, где строки представляют объекты, а столбцы пользователей; значения в ячейках матрицы определяют оценки или действия пользователей по отношению к объектам;
	\item[2.] Вычисление сходства между пользователями;
	\item[3.] Генерация рекомендаций с опорой на сходство между объектами, которые пользователь уже оценил. 
\end{enumerate}
\pagebreak