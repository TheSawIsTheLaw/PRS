\section*{ЗАКЛЮЧЕНИЕ}
\addcontentsline{toc}{section}{ЗАКЛЮЧЕНИЕ}
В ходе выполнения работы было проведено сравнение алгоритмов коллаборативной фильтрации по пользователю и по объекту.

Были решены следующие задачи:
\begin{itemize}
	\item приведено описание алгоритмов;
	\item приведено описание используемых для исследования данных;
	\item проведено сравнение алгоритмов по времени работы и точности предсказания.
\end{itemize}

Проведенные исследования показали, что на использованном датасете с точки зрения ROC-AUC в двух вариациях лучше всего себя показывает фильтрация с параметром используемых похожих объектов, лежащим в интервале от 3 до 5, и используемой мерой Жаккара. Также стало понятно, что некоторый выигрыш по времени за счет использования различных мер близости, можно получить лишь при фильтрации по пользователю.

\pagebreak