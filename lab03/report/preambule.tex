%----------------------- Преамбула -----------------------
\documentclass[14pt, a4paper]{extarticle}

\usepackage{extsizes} % Для добавления в параметры класса документа 14pt

% Для работы с несколькими языками и шрифтом Times New Roman по-умолчанию
\usepackage{cmap} % Улучшенный поиск русских слов в полученном pdf-файле
\usepackage[english,russian]{babel}
\usepackage[T2A]{fontenc} % Поддержка русских букв
\usepackage[utf8]{inputenc} % Кодировка utf8
\usepackage{fontspec}
\setmainfont{Times New Roman}
\usepackage{float}

%CERF
\usepackage{array}
\newcolumntype{C}[1]{>{\centering\arraybackslash}p{#1}}

% ГОСТовские настройки для полей и абзацев
\usepackage{geometry}
\geometry{left=30mm}
\geometry{right=15mm}
\geometry{top=20mm}
\geometry{bottom=20mm}
\usepackage{misccorr}
\usepackage{indentfirst}
\usepackage{enumitem}
\setlength{\parindent}{1.25cm}
%\setlength{\parskip}{1em} % поменять
%\linespread{1.3}
\renewcommand{\baselinestretch}{1.5}
\setlist{nolistsep} % Отсутствие отступов между элементами \enumerate и \itemize

% Дополнительное окружения для подписей
\usepackage{array}
\newenvironment{signstabular}[1][1]{
	\renewcommand*{\arraystretch}{#1}
	\tabular
}{
	\endtabular
}

% Переопределение стандартных \section, \subsection, \subsubsection по ГОСТу;
% Переопределение их отступов до и после для 1.5 интервала во всем документе
\usepackage{titlesec}

\titleformat{\section}[block]
{\bfseries\normalsize\filcenter}{\thesection}{1em}{}

\titleformat{\subsection}[hang]
{\bfseries\normalsize}{\thesubsection}{1em}{}
\titlespacing\subsection{\parindent}{\parskip}{\parskip}

\titleformat{\subsubsection}[hang]
{\bfseries\normalsize}{\thesubsubsection}{1em}{}
\titlespacing\subsubsection{\parindent}{\parskip}{\parskip}

% Работа с изображениями и таблицами; переопределение названий по ГОСТу
\usepackage{float}
\usepackage{setspace}
\onehalfspacing % Полуторный интервал
\frenchspacing
\usepackage{caption}
\usepackage{indentfirst} % Красная строка
\usepackage{titlesec}
\titleformat{\chapter}{\LARGE\bfseries}{\thechapter}{20pt}{\LARGE\bfseries}
\titleformat{\section}{\centering\Large\bfseries}{\thesection.\quad}{20pt}{\Large\bfseries}
\usepackage{listings}

% Выравнивание "где" по ГОСТу для формул
\usepackage{eqexpl}

\usepackage{graphicx}
\usepackage{slashbox} % Диагональное разделение первой ячейки в таблицах

% Цвета для гиперссылок и листингов
\usepackage{color}

% Для нормальной нумерации картинок, разделённой по section'ам
\usepackage{chngcntr}
\counterwithin{figure}{section}

% Гиперссылки \toc с кликабельностью
\usepackage{hyperref}

\hypersetup{
	linktoc=all,
	linkcolor=black,
	colorlinks=true,
}

% Листинги
\setsansfont{Arial}
\setmonofont{Courier New}

\usepackage{color} % Цвета для гиперссылок и листингов
%\definecolor{comment}{rgb}{0,0.5,0}
%\definecolor{plain}{rgb}{0.2,0.2,0.2}
%\definecolor{string}{rgb}{0.91,0.45,0.32}
%\hypersetup{citecolor=blue}
\hypersetup{citecolor=black}

\usepackage{listings}

% Для листинга кода:
\lstset{
	basicstyle=\footnotesize\ttfamily,
	language=C, % Или другой ваш язык -- см. документацию пакета
	commentstyle=\color{comment},
	numbers=left,
	numberstyle=\tiny\color{plain},
	numbersep=5pt,
	tabsize=4,
	extendedchars=\true,
	breaklines=true,
	breakatwhitespace=false,
	keywordstyle=\color{blue},
	frame=b,
	stringstyle=\ttfamily\color{string}\ttfamily,
	showspaces=false,
	showtabs=false,
	xleftmargin=17pt,
	framexleftmargin=17pt,
	framexrightmargin=5pt,
	framexbottommargin=4pt,
	showstringspaces=false,
	inputencoding=utf8x,
	keepspaces=true
}

\DeclareCaptionLabelSeparator{line}{\ --\ }
\DeclareCaptionFont{white}{\color{white}}
\DeclareCaptionFormat{listing}{\colorbox[cmyk]{0.43,0.35,0.35,0.01}{\parbox{\textwidth}{\hspace{15pt}#1#2#3}}}
\captionsetup[lstlisting]{
	format=listing,
	labelfont=white,
	textfont=white,
	singlelinecheck=false,
	margin=0pt,
	font={bf,footnotesize},
	labelsep=line
}

\usepackage{ulem} % Нормальное нижнее подчеркивание
\usepackage{hhline} % Двойная горизонтальная линия в таблицах
\usepackage[figure,table]{totalcount} % Подсчет изображений, таблиц
\usepackage{rotating} % Поворот изображения вместе с названием
\usepackage{lastpage} % Для подсчета числа страниц

\makeatletter
\renewcommand\@biblabel[1]{#1.}
\makeatother

\renewcommand\bibname{}

\usepackage{color}
\usepackage[cache=false, newfloat]{minted}
\newenvironment{code}{\captionsetup{type=listing}}{}
\SetupFloatingEnvironment{listing}{name=Листинг}

\usepackage{amsmath}
\usepackage{slashbox}