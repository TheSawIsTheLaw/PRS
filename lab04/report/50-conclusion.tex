\section*{ЗАКЛЮЧЕНИЕ}
\addcontentsline{toc}{section}{ЗАКЛЮЧЕНИЕ}
В ходе выполнения работы было проведено сравнение алгоритмов коллаборативной фильтрации по пользователю и по объекту.

Были решены следующие задачи:
\begin{itemize}
	\item приведено описание алгоритмов;
	\item приведено описание используемых для исследования данных;
	\item приведены зависимости скорости работы алгоритмов от заданных параметров.
\end{itemize}

В результате проведенных исследований стало известно, что SVD++ с включенным кэшированием на заданном датасете работает заметно быстрее, чем Funk SVD, и при изменении значения параметра регуляризации, так и при изменении значения параметра скорости обучения.

Также SVD++ при изменении параметра регуляризации показывает метрики RMSE и MAE меньше Funk SVD, однако при изменении параметра скорости обучения на значениях $\approx 0.16$ для RMSE и $\approx 0.20$ для MAE он начинает уступать по точности Func SVD.

\pagebreak