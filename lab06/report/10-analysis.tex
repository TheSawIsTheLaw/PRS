\section{Аналитический раздел}

\subsection{Алгоритмы выявления сообществ}

Алгоритмы выяления сообществ -- это методы анализа сетей, направленные на выделение групп узлов в сети, которые тесно взаимодействуют друг с другом. Сообщества в сетях обычно определяются на основе структуры графа взаимосвязей между узлами.

\subsection{Label Propogation Communities}

LPC или алгоритм распространения меток основан на распространении информации через сеть. Этот алгоритм часто используется для выявления сообществ в графах, где узлы схожи между собой.

Основные шаги алгоритма:

\begin{enumerate}
	\item[1.] Инициализация меток -- каждый узел в графе ининциализируется с некоторой начальной меткой, это может быть уникальный идентификатор или любая другая информация;
	\item[2.] Распространение меток -- метки начинают распространяться по сети, на каждом шаге узлы обновляют свои метки на основе меток соседей; Чаще всего узел принимает метку, которая преобладает среди его соседей;
	\item[3.] Итерирование -- шаги распространения повторяются до тех пор, пока метки не стабилизируются или до достижения максимального числа итераций;
	\item[4.] Финализация -- после завершения итераций, каждый узел устанавливает свою финальную метку, которая была определена в результате многократного распространения и обновления.
\end{enumerate}

Алгоритм не всегда сходится к однозначному результату, а финальный метки могут сильно зависеть от начальной конфигурации.

\subsection{Алгоритм Лювина}

Данный алгоритм основан на оптимизации меры модулярности, которая измеряет качество разбиения графа на сообщества.

Основные шаги алгоритма:

\begin{enumerate}
	\item[1.] Инициализация -- с самого начала каждый узел считается отдельным сообществом;
	\item[2.] Оптимизация модулярности -- алгоритм производит попытки объединять сообщества с целью максимизации модулярности графа (модулярность -- мера качества разбиения графа на сообщества, оценивает, насколько хорошо внутригрупповые связи сильнее, чем случайные связи между узлами);
	\item[3.] Шаги объединения и переразбиения -- алгоритм проходит через граф несколько раз, выполняя два основных шага: объединение и переразбиение; На шаге объединения узлы объединяются в еще более крупные сообщества, а на шаге переразбиения происходит оптимизация внутригрупповых связей путем перемещения узлом между сообществами;
	\item[4.] Оптимизация модулярности -- после каждого объединения или переразбиения происходит оптимизация модулярности, чтобы убедиться, что изменение соответствует улучшению структуры графа;
	\item[4.1] Шаги объединения и переразбиения производятся до тех пор, пока модулярность не перестанет увеличиваться либо же по достижению заданного числа итераций. 
\end{enumerate}

Алгоритм Лювена известен своей способностью быстро выявлять сообщества в крупных сетях.

\subsection{Fluid Communities}

Данный алгоритм основан на идее введения ряда флюидов (то есть сообществ) в неоднородную среду (то есть в графе, не являющимся полным графом), где флюиды будут расширяться и воздействовать друг на друга под влиянием топологии среды до достижения устойчивого состояния.

К достоинствам алгоритма относят его асинхронность, возможность задать количество обнаруживаемых сообществ путем определения начального количества флюидов, а также избегание создания больших сообществ.

\pagebreak